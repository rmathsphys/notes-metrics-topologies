%!TEX root = ../main.tex
\section{Connectedness}
\subsection{Preliminary definitions}
\begin{ndfn}[Partition]
  Take $A, B \subseteq X$. We say that the pair $(A,B)$ is a partition of $X$ if $X = A \cup B$ and $A \cap B = \emptyset$.
\end{ndfn}
If $(A,B)$ is a partition of $X$, then we say that $A$ and $B$ partition $X$.

\begin{ndfn}[Connected]
  A topological space $X$ is called connected if the only partitions of $X$ using open sets are $(X, \emptyset)$, $(\emptyset, X)$.
\end{ndfn}
A topological space that is not connected, is called disconnected.

\begin{remark}
  Let $(A,B)$ be a partition of $X$. Both $A$ and $B$ are open $\iff$ both $A$ and $B$ are closed.
\end{remark}

\begin{ndfn}
  A subset $S \subseteq X$ is connected (disconnected) if $S$ is connected (disconnected) as a topological space with the subspace topology.
\end{ndfn}

\begin{nthm}\label{thm:disconnected-equiv}
  The following statements are equivalent:
  \begin{enumerate}
  \item $X$ is disconnected.
  \item $X$ has a partition of two non-empty open sets.
  \item $X$ has a partition of two non-empty closed sets.
  \item $X$ has a subset other than $\emptyset$ and $X$ that is both open and closed.
  \item There is a continuous surjection $f: X \to \set{0,1}$, where $\set{0,1}$ has the discrete topology.
  \end{enumerate}
\end{nthm}

In (v), the co-domain can be any set with exactly two distinct elements; the choice $\set{0,1}$ is arbitrary.

(v) can be used to show that $X$ is connected by showing that any continuous $f: X \to \set{0,1}$ must be constant.

\begin{ndfn}
  We say that $S \subseteq X$ is separated by open sets $U, V$ in $X$ if
  \begin{equation*}
    S \subseteq U \cup V,
    \quad U \cap V \cap S = \emptyset,
    \quad U \cap S \neq \emptyset,
    \quad V \cap S \neq \emptyset.
  \end{equation*}
\end{ndfn}

\begin{nprop}
  $S \subseteq X$ is disconnected if and only if $S$ is separated by some open sets $U, V$ in $X$.
\end{nprop}

An interval in $\R$ is one of the following sets
\begin{itemize}
\item $\emptyset$
\item $\set{x}$, for any $x \in \R$
\item $[a,b] \equiv \set{x \in \R \st a \leq x \leq b}$
\item $[a,b) \equiv \set{x \in \R \st a \leq x \leq b}$. $b=\infty$ is allowed.
\item $(a,b] \equiv \set{x \in \R \st a \leq x \leq b}$. $a=-\infty$ is allowed.
\item $(a,b) \equiv \set{x \in \R \st a \leq x \leq b}$. $a=-\infty$ and $b=\infty$ is allowed.
\end{itemize}

\begin{nlemma}
  $S \subseteq \R$ is connected if an only if it is an interval.
\end{nlemma}
This suggests that an interval can simply be defined as a connected subset of $\R$.

\subsection{Operations on connected sets}
\begin{nprop}\label{thm:connected-check-a}
  Suppose that $C_j$ ($j \in J$) are connected subsets of $X$, and $C_i \cap C_j \neq \emptyset$ for every $i,j$. Then,
  \begin{equation*}
    K = \bigcap_{j \in J} C_j
  \end{equation*}
  is also a connected subset of $X$.
\end{nprop}

\begin{nprop}\label{thm:connected-check-b}
  Suppose that $C_1$ and $C_2$ are connected subsets of $X$ and $\closure{C_1} \cap C_2 \neq \emptyset$. Then, $C_1 \cup C_2$ is connected.
\end{nprop}

\begin{nthm}\label{thm:connected-check-c}
  Suppose that $C$ and $C_j$ ($j \in J$) are connected subsets of $X$, and $C_j \cap \closure{C} \neq \emptyset$. Then,
  \begin{equation*}
    K = C \cup \left(\bigcup_{j \in J} C_j\right)
  \end{equation*}
  is connected.
\end{nthm}

\begin{ncor}\label{thm:connected-check-d}
  If $C \subset T$ is connected, then any set $K$ satisfying $C \subset K \subset \closure{C}$ is also connected.
\end{ncor}
\begin{proof}
  We have $K = C \cup \bigcup_{x \in K} \set{x}$ and $\set{x} \cap \closure{C} \neq \emptyset$ for each $x \in K$.
\end{proof}

\begin{nthm}\label{thm:connected-invariant}
  The continuous image of a connected set is connected.
\end{nthm}
\begin{proof}
  Let $f: X \to Y$ be a continuous map, and suppose $X$ is connected. If $f(X)$ is not connected then there exists a surjective continuous map $g: f(X) \to \set{0,1}$. But then $g \circ f: X \to \set{0,1}$ is continuous and surjective. By theorem (\ref{thm:disconnected-equiv}), this implies that $X$ is disconnected, which is a contradiction. So, $f(X)$ is connected.
\end{proof}
This proves that connectedness is a topological invariant.

\begin{nthm}\label{thm:connected-product}
  Let $X$ and $Y$ be connected topological spaces. Then, $X \times Y$ is also connected, using the product topology.
\end{nthm}

\begin{negg}
  Some examples (EXPAND SEPARATELY):
  \begin{itemize}
  \item $\R^2 = \R \times \R$ is connected. (By theorem (\ref{thm:connected-product}).)
  \item Any circle in $\R$ is connected. (It is a continuous image of an interval in $\R$.)
  \item The `topologist's sine curve' $\mathcal{S}$ is connected.
  \begin{equation*}
    \mathcal{S} = \set*{\left(x, \sin\frac{1}{x}\right) \st x \in \R, x \neq 0} \cup \set{(0,0)}.
  \end{equation*}
  \item The `harmonic comb'
  \begin{equation*}
    \set{(x,y) \set y=0, x \in (0,1]} \cup \set{(1/n, y) \set n \in \N, y \in [0,1]} \cup \set{(0,1)}
  \end{equation*}
  is connected.\eggqed
  \end{itemize}
\end{negg}

Theorem (\ref{thm:connected-invariant}) already established that connectedness is a topological invariant. For practical purposes, a more useful invariant is the following property: `$X-\set{x}$ is connected for every $x \in X$'.

\begin{negg}
  We can use this property to prove that certain topological spaces are not homeomorphic.
  \begin{itemize}
  \item $[0,1]$ is not homeomorphic to the circle $S^1$: $[0,1/2)\cup(1/2,1]$ is disconnected, but with an arbitrary point removed the circle is still connected.

  \item $\R$ is not homeomorphic to $\R^2$: $(-\infty,0)\cup(0,\infty)$ is disconnected, but $\R^2 - \set{x}$ is still connected for every $x \in \R^2$.\eggqed
  \end{itemize}
\end{negg}

\subsection{Path-connected spaces}
\begin{ndfn}[Path]
  Take $a,b \in X$. A path (in $X$) from $a$ to $b$ is a continuous map $\varphi: [0,1] \to X$ such that $\varphi(0)=a$ and $\varphi(1)=b$.
\end{ndfn}

\begin{ndfn}[Path-connected]
  A topological space $X$ is called path-connected if every pair of points in $X$ can be joined using a path in $X$.
\end{ndfn}

\begin{nthm}\label{thm:connected-path-connected}
  A path-connected space is connected.
\end{nthm}
\begin{proof}
  Fix some $a \in X$, and consider any $b \in X$. Then, there is a path $\varphi: [0,1] \to X$. The Let $C_b = \varphi([0,1])$. This is connected because $\varphi$ is continuous and $[0,1]$ is connected.

  Then, $X = \set{a} \cup \bigcup_{b \in X} C_b$, and each $C_b$ contains $a$. So, by theorem (\ref{thm:connected-check-c}), $X$ is connected.
\end{proof}

The converse of this theorem is not true in general. For example, the `harmonic comb' and the `topologist's sine curve' studied earlier are connected but not path connected.

\begin{nprop}
  Connected open sets of $\R^n$ are path connected.
\end{nprop}
