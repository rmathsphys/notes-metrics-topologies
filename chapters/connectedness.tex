%!TEX root = ../main.tex
\section{Connectedness}
\begin{ndfn}[Partition]
  Take $A, B \subseteq X$. We say that the pair $(A,B)$ is a partition of $X$ if $X = A \cup B$ and $A \cap B$.
\end{ndfn}
If $(A,B)$ is a partition of $X$, then we say that $A$ and $B$ partition $X$.

\begin{ndfn}[Connected]
  A topological space $X$ is called connected if the only partition of $X$ using open sets are $(X, \set{})$, $(\set{}, X)$.
\end{ndfn}
A topological space that is not connected, is called disconnected.

\begin{remark}
  Let $(A,B)$ be a partition of $X$. Both $A$ and $B$ are open $\iff$ both $A$ and $B$ are closed.
\end{remark}

\begin{ndfn}
  A subset $S \subseteq X$ is connected (disconnected) if $S$ is connected (disconnected) as a topological space with the subspace topology.
\end{ndfn}

\begin{nthm}
  The following statements are equivalent:
  \begin{enumerate}
  \item $X$ is disconnected.
  \item $X$ has a partition of two non-empty open sets.
  \item $X$ has a partition of two non-empty closed sets.
  \item $X$ has a subset other than $\emptyset$ and $X$ that is both open and closed.
  \item There is a continuous surjection $f: X \to \set{0,1}$, where $\set{0,1}$ has the discrete topology.
  \end{enumerate}
\end{nthm}

In (v), the co-domain can be any set with exactly two distint elements; the choice $\set{0,1}$ is arbitrary.

(v) can be used to show that $X$ is connected by showing that any continuous $f: X \to \set{0,1}$ must be constant.

\begin{ndfn}
  We say that $S \subseteq X$ is separated by open sets $U, V$ in $X$ if
  \begin{equation*}
    S \subseteq U \cup V,
    \quad U \cap V \cap S = \emptyset,
    \quad U \cap S \neq \emptyset,
    \quad V \cap S \neq \emptyset.
  \end{equation*}
\end{ndfn}

\begin{nprop}
  $S \subseteq X$ is disconnected if and only if $S$ is separated by some open sets $U, V$ in $X$.
\end{nprop}

An interval in $\R$ is one of the following sets
\begin{itemize}
\item $\emptyset$
\item $\set{x}$, for any $x \in \R$
\item $[a,b] \equiv \set{x \in \R \st a \leq x \leq b}$
\item $[a,b) \equiv \set{x \in \R \st a \leq x \leq b}$. $b=\infty$ is allowed.
\item $(a,b] \equiv \set{x \in \R \st a \leq x \leq b}$. $a=-\infty$ is allowed.
\item $(a,b) \equiv \set{x \in \R \st a \leq x \leq b}$. $a=-\infty$ and $b=\infty$ is allowed.
\end{itemize}

\begin{nlemma}
  $S \subseteq \R$ is connected if an only if it is an interval.
\end{nlemma}
This suggests that an interval can simply be defined as a connected subset of $\R$.